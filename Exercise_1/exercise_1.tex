\subsection{Create a table of descriptive statistics of the variables in the dataset.}
% latex table generated in R 4.2.1 by xtable 1.8-4 package
% Wed Dec  6 12:37:06 2023
\begin{table}[ht]
\centering
\begin{tabular}{llllllr}
  \hline
var & median & mean & min & max & sd & NAs \\ 
  \hline
agro\_emp & 18.6 & 25.1 & 0.1 & 86.3 & 22.3 &  30 \\ 
  bribery & 11.7 & 17.0 & 0.0 & 67.1 & 14.7 &  87 \\ 
  gfce & 16.5 & 17.7 & 5.1 & 62.9 & 8.4 &  36 \\ 
  literacy & 93.0 & 83.6 & 24.2 & 100.0 & 19.3 &  61 \\ 
  log\_gdp & 9.4 & 9.3 & 6.6 & 11.6 & 1.2 &  22 \\ 
  pop\_total & 6.2e+06 & 3.4e+07 & 1.1e+04 & 1.4e+09 & 1.4e+08 &   2 \\ 
  self\_emp & 35.0 & 40.9 & 0.4 & 94.8 & 27.0 &  30 \\ 
  stocks & 6.4 & 28.8 & 0.0 & 538.7 & 66.8 & 131 \\ 
  sample\_size & 715.1 & 3.6e+03 & 120.1 & 1.4e+05 & 1.3e+04 &   2 \\ 
   \hline
\end{tabular}
\caption{Descriptive statistics} 
\label{desc}
\end{table}



The data is made of 217 countries and certain variables contains a significant
amount of missing values (NAs) (See \ref{desc}). Bribery is the worst variable with up to 87 NAs. Using
this variable in a regression would yield a low precision as the size of the sample is quite small.
This table demonstrates how different countries are, indeed the min-max interval is rather close to one for rate variables and the standart deviations (sd)
are very significant compared to the mean value of each variable.

\stepcounter{subsection}
\subsubsection{Is it the case that self-employment is correlated with how rich a country is (in-terms of log GDP per-capita)?}


\begin{figure}[htbp]
  \centering
  \includegraphics[width=0.8\textwidth]{Exercise_1/OUTPUT/emp_wrt_gdp.png}
  \caption{Self employment rate with respect to GDP}
  \label{fig:emp_gdp}
\end{figure}


Figure \ref{fig:emp_gdp} shows a clear negative relationship between the share of self employment and gdp. The empirical correlation coefficient is equal to $-0.89$
which is very close to $-1$. The anticorrelation is very strong.
\subsubsection{Is it the case that countries with higher share of employment in agriculture also have higher self-employment rates?}

\begin{figure}[htbp]
  \centering
  \includegraphics[width=0.8\textwidth]{Exercise_1/OUTPUT/emp_wrt_agro.png}
  \caption{Self employment rate with respect to employment share in the agricultural sector}
  \label{fig:emp_agro}
\end{figure}

As in the previous question, Figure \ref{fig:emp_agro} shows a clear positive relationship between the share of self employment and share of employment in the agricultural sector. The empirical correlation coefficient is equal to $0.91$
which is very close to $1$. The correlation is very strong.

\subsubsection{Present a bar graph comparing the mean self-employment rates in each of these 3 literacy-based categories of countries.}
\begin{figure}[htbp]
  \centering
  \includegraphics[width=0.8\textwidth]{Exercise_1/OUTPUT/bar_emp_literacy.png}
  \caption{Average self employement rate as a function of the literacy category}
  \label{fig:emp_lit}
\end{figure}

Figure \ref{fig:emp_lit} seems to demonstrate a negative relationship between the literacy category of the population and the self employment rate.

\subsection{Estimate the model parameters described using OLS, report your results and summarize them}

% Table created by stargazer v.5.2.3 by Marek Hlavac, Social Policy Institute. E-mail: marek.hlavac at gmail.com
% Date and time: Jeu, déc 07, 2023 - 11:24:15
% Requires LaTeX packages: dcolumn 
\begin{table}[!htbp] \centering 
  \caption{Linear Regressions - Exercise 1} 
  \label{results_1} 
\begin{tabular}{@{\extracolsep{5pt}}lD{.}{.}{-3} D{.}{.}{-3} } 
\\[-1.8ex]\hline 
\hline \\[-1.8ex] 
 & \multicolumn{2}{c}{\textit{Dependent variable:}} \\ 
\cline{2-3} 
\\[-1.8ex] & \multicolumn{2}{c}{self\_emp} \\ 
\\[-1.8ex] & \multicolumn{1}{c}{(1)} & \multicolumn{1}{c}{(2)}\\ 
\hline \\[-1.8ex] 
 log\_gdp & -6.506^{***} & -5.520 \\ 
  & (1.755) & (4.042) \\ 
  & & \\ 
 literacy & -0.313^{***} & -0.358^{**} \\ 
  & (0.070) & (0.175) \\ 
  & & \\ 
 agro\_emp & 0.592^{***} & 0.628^{***} \\ 
  & (0.080) & (0.176) \\ 
  & & \\ 
 gfce &  & -0.922^{**} \\ 
  &  & (0.380) \\ 
  & & \\ 
 stocks &  & 0.110^{*} \\ 
  &  & (0.061) \\ 
  & & \\ 
 bribery &  & -0.111 \\ 
  &  & (0.156) \\ 
  & & \\ 
 Constant & 113.219^{***} & 121.953^{***} \\ 
  & (16.361) & (41.265) \\ 
  & & \\ 
\hline \\[-1.8ex] 
Observations & \multicolumn{1}{c}{143} & \multicolumn{1}{c}{49} \\ 
R$^{2}$ & \multicolumn{1}{c}{0.845} & \multicolumn{1}{c}{0.828} \\ 
Adjusted R$^{2}$ & \multicolumn{1}{c}{0.841} & \multicolumn{1}{c}{0.804} \\ 
Residual Std. Error & \multicolumn{1}{c}{10.574 (df = 139)} & \multicolumn{1}{c}{9.262 (df = 42)} \\ 
F Statistic & \multicolumn{1}{c}{252.152$^{***}$ (df = 3; 139)} & \multicolumn{1}{c}{33.720$^{***}$ (df = 6; 42)} \\ 
\hline 
\hline \\[-1.8ex] 
\textit{Note:}  & \multicolumn{2}{r}{$^{*}$p$<$0.1; $^{**}$p$<$0.05; $^{***}$p$<$0.01} \\ 
\end{tabular} 
\end{table} 

Denoting this linear model as (1) and estimating it yields the results displayed in Table \ref{results_1} (first column).
This OLS estimation was based on only 143 observations, which is far from 217. This raises a
question, are NAs equally distributed across the sample ? That being aknowledged, this regression seems
to be very significant. The overall significance is very high as the F-stat demonstrates.
Every of the three coefficients are significantly different from zero at the 1\% level.
The sign of the variables is in line with the previous discussion in question 1.1.
According to this estimation, a 1\% increase in GDP corresponds to a 6.5\% decrease in the share of self employement.
A 1\% increase in the literacy rate is associated with a 0.3\% decrease in the dependent variable, while a 0.59\% increase in self employment shares 
can be explained by a 1\% increase in the share of the agricultural sector.

\subsection{Describe how you could estimate $\beta_3$ using a 3-step procedure based on the Frisch-Waugh Theorem.}
The Frisch-Waugh theorem can be used for estimating specific coefficients in a multiple regression model while controlling for other variables. Here are the simplified three steps:


\textbf{Step 1:}\\
Identify control variables, here log\_gdp and literacy and regress it on the dependent variable to get the residuals $\hat{r}$.
\begin{equation*}
  \text{self\_emp} = \alpha_0 + \alpha_1 log\_gdp + \alpha_2 literacy + r 
\end{equation*}
\textbf{Step 2:}\\
Regress the variable of interest on the control variables and get the residuals $\hat{u}$.
\begin{equation*}
  \text{agro\_emp} = \gamma_0 + \gamma_1 log\_gdp + \gamma_2 literacy + u 
\end{equation*}
\textbf{Step 3:}\\
Regress $\hat{r}$ on $\hat{u}$ and get the coefficient of interest $\beta_3$.
\begin{equation*}
  \hat{r} = \beta_0 + \beta_3 \hat{u} + \epsilon
\end{equation*}
The coefficient $\beta_3$ is the estimate of the parameter for agro\_emp after controlling for log\_gdp and literacy.

\subsection{Implement this 3-step procedure and compare your estimate and standard error of $\beta_3$}